\documentclass[12pt,letterpaper]{article}
\usepackage{graphicx,textcomp}
\usepackage{natbib}
\usepackage{setspace}
\usepackage{fullpage}
\usepackage{color}
\usepackage[reqno]{amsmath}
\usepackage{amsthm}
\usepackage{fancyvrb}
\usepackage{amssymb,enumerate}
\usepackage[all]{xy}
\usepackage{endnotes}
\usepackage{lscape}
\newtheorem{com}{Comment}
\usepackage{float}
\usepackage{hyperref}
\newtheorem{lem} {Lemma}
\newtheorem{prop}{Proposition}
\newtheorem{thm}{Theorem}
\newtheorem{defn}{Definition}
\newtheorem{cor}{Corollary}
\newtheorem{obs}{Observation}
\usepackage[compact]{titlesec}
\usepackage{dcolumn}
\usepackage{tikz}
\usetikzlibrary{arrows}
\usepackage{multirow}
\usepackage{xcolor}
\newcolumntype{.}{D{.}{.}{-1}}
\newcolumntype{d}[1]{D{.}{.}{#1}}
\definecolor{light-gray}{gray}{0.65}
\usepackage{url}
\usepackage{listings}
\usepackage{color}

\definecolor{codegreen}{rgb}{0,0.6,0}
\definecolor{codegray}{rgb}{0.5,0.5,0.5}
\definecolor{codepurple}{rgb}{0.58,0,0.82}
\definecolor{backcolour}{rgb}{0.95,0.95,0.92}

\lstdefinestyle{mystyle}{
	backgroundcolor=\color{backcolour},   
	commentstyle=\color{codegreen},
	keywordstyle=\color{magenta},
	numberstyle=\tiny\color{codegray},
	stringstyle=\color{codepurple},
	basicstyle=\footnotesize,
	breakatwhitespace=false,         
	breaklines=true,                 
	captionpos=b,                    
	keepspaces=true,                 
	numbers=left,                    
	numbersep=5pt,                  
	showspaces=false,                
	showstringspaces=false,
	showtabs=false,                  
	tabsize=2
}
\lstset{style=mystyle}
\newcommand{\Sref}[1]{Section~\ref{#1}}
\newtheorem{hyp}{Hypothesis}

\usepackage[utf8]{inputenc}
\usepackage[T1]{fontenc}


\title{Problem Set 1}
\date{\today}
\author{Hanyu Li (Student ID: 25346841)}

\begin{document}
	\maketitle
	
	\section*{Question 1: Education}
	
	A school counselor was curious about the average of IQ of the students in her school and took a random sample of 25 students' IQ scores. The following is the data set:
	
	\vspace{.5cm}
	
	\lstinputlisting[language=R, firstline=38, lastline=38]{PS01_answers_HanyuLi_25346841.R}  
	
	\vspace{1cm}
	
	\begin{enumerate}
		\item Find a 90\% confidence interval for the average student IQ in the school.\\
		
		\lstinputlisting[language=R, firstline=40, lastline=58]{PS01_answers_HanyuLi_25346841.R} 
		
		The 90\% confidence interval is (93.95993, 102.92007); sample mean (98.44).
		
		\item Next, the school counselor was curious whether the average student IQ in her school is higher than the average IQ score (100) among all the schools in the country.\\ 
		
		\noindent Using the same sample, conduct the appropriate hypothesis test with $\alpha=0.05$.
	\end{enumerate}
	
	\lstinputlisting[language=R, firstline=61, lastline=64]{PS01_answers_HanyuLi_25346841.R}
	
	\begin{verbatim}
		data:  y
		t = -0.59574, df = 24, p-value = 0.7215
		alternative hypothesis: true mean is greater than 100
		95 percent confidence interval:
		93.95993      Inf
		sample estimates:
		mean of x 
		98.44 
	\end{verbatim} 
	
	The outcome shows: t value is rather close to 0, indicating there is no apparent difference between the observed mean and 100 and p-value is obviously greater than 0.05, which means the null hypothesis can't be rejected. In other words, the average student IQ in the school can't be seen as higher than the average IQ score (100) among all the schools in the country.
	
	\newpage
	
	\section*{Question 2: Political Economy}
	
	\noindent Researchers are curious about what affects the amount of money communities spend on addressing homelessness. The following variables constitute our data set about social welfare expenditures in the USA.
	
	\vspace{.5cm}
	
	\begin{tabular}{r|l}
		\texttt{State} & 50 states in US \\
		\texttt{Y} & per capita expenditure on shelters/housing assistance in state\\
		\texttt{X1} & per capita personal income in state \\
		\texttt{X2} & Number of residents per 100,000 that are ``financially insecure'' in state\\
		\texttt{X3} & Number of people per thousand residing in urban areas in state \\
		\texttt{Region} & 1=Northeast, 2= North Central, 3= South, 4=West \\
	\end{tabular}
	
	\vspace{.5cm}
	\noindent Explore the \texttt{expenditure} data set and import data into \texttt{R}.
	
	\vspace{.5cm}
	
	\begin{itemize}
		
		\item
		Please plot the relationships among \emph{Y}, \emph{X1}, \emph{X2}, and \emph{X3}? What are the correlations among them (you just need to describe the graph and the relationships among them)?
		
		\vspace{.5cm}
		\lstinputlisting[language=R, firstline=78, lastline=86]{PS01_answers_HanyuLi_25346841.R}
		
	The correlations between Y,X1,X2 and X3 are shown as below:
	\begin{verbatim}
	           Y        X1        X2        X3
	Y  1.0000000 0.5317212 0.4482876 0.4636787
	X1 0.5317212 1.0000000 0.2056101 0.5952504
	X2 0.4482876 0.2056101 1.0000000 0.2210149
	X3 0.4636787 0.5952504 0.2210149 1.0000000
	\end{verbatim}
		
  \begin{figure}[H]
	\centering
	\includegraphics[width=.85\textwidth]{plot_all.pdf}
	\caption{\footnotesize Relationships Among Y, X1, X2, X3}
	\label{fig:plot_1}
   \end{figure}
	
	
	Figure \ref{fig:plot_1} and correlation analysis outcomes indicates that:\\
	A moderate and positive correlation between per capita expenditure on shelters/housing assistance (Y) and per capita personal income (X1), with a correlation coefficient of approximately 0.53.\\
	Y also shows moderate positive associations with both financial insecurity (X2, r$\approx$0.45) and urban population density (X3, r$\approx$0.46).\\
	Among the independent variables, per capita personal income (X1) and urban population density (X3) exhibited a strong positive correlation (r$\approx$0.60), while the relationships involving financial insecurity (X2) and the other predictors were found to be weaker.
	
	\newpage
		
		\item
		Please plot the relationship between \emph{Y} and \emph{Region}? On average, which region has the highest per capita expenditure on housing assistance?
		
		\vspace{.5cm}
		\lstinputlisting[language=R, firstline=94, lastline=105]{PS01_answers_HanyuLi_25346841.R}
		\vspace{.2cm}
		
	    \begin{figure}[H]
		  \centering
		  \includegraphics[width=.85\textwidth]{plot_Y_RG.pdf}
		  \caption{\footnotesize Relationship Between Y And Region}
		  \label{fig:plot_2}
	    \end{figure}
	  
	  \vspace{.2cm}
	  Figure \ref{fig:plot_2} shows averagely, west region has the highest per capita expenditure on housing assistance.
		
   	  \vspace{1cm} 
   	  
	\newpage
		
		\item
		Please plot the relationship between \emph{Y} and \emph{X1}? Describe this graph and the relationship. Reproduce the above graph including one more variable \emph{Region} and display different regions with different types of symbols and colors.
		
		\lstinputlisting[language=R, firstline=109, lastline=114]{PS01_answers_HanyuLi_25346841.R}
		

		\begin{figure}[H]
			\centering
			\includegraphics[width=.85\textwidth]{plot_Y_X1.pdf}
			\caption{\footnotesize Relationship Between Y And X1}
			\label{fig:plot_3}
		\end{figure}
		
		Figure \ref{fig:plot_3} indicates that per capita expenditure (Y) is positively associated with per capita personal income (X1), which means as the state's per capita personal income increases, shelters/housing assistance spending per capita grows as well.
		
		\vspace{1cm}
		
		Adding variable \emph{Region} to the plot and display different regions with different types of symbols and colors.
		\lstinputlisting[language=R, firstline=117, lastline=124]{PS01_answers_HanyuLi_25346841.R}
		
		\begin{figure}[H]
			\centering
			\includegraphics[width=.85\textwidth]{plot_Y_X1_byRG.pdf}
			\caption{\footnotesize Relationship Between Y And X1 By Region}
			\label{fig:plot_4}
		\end{figure}
		
		
	\end{itemize}
	
\end{document}
